\documentclass[12pt]{article}

\usepackage{amsmath}
\usepackage{hyperref}
\usepackage{listings}
\usepackage{palatino}

\lstset{basicstyle=\ttfamily,breaklines=true}

\DeclareMathOperator{\Cov}{Cov}
\DeclareMathOperator{\Var}{Var}

\title{Code to estimate a permanent-transitory income process}
\author{Jonathan Shaw}
\date{\today}

\begin{document}
\maketitle

\section{Introduction}

This note describes accompanying code to estimate a permanent-transitory income process using the generalised method of moments (GMM).

\section{Population moments}

The income process we consider is the following:
%
\begin{align}
	y_{it} &= u_{it} + v_{it} \\
    u_{it} &= u_{i,t-1} + w_{it} \\
    v_{it} &= e_{i,t-1} + \theta e_{it}
\end{align}
%
where \(y_{it}\) is log income, \(u_{it}\) is a permanent (random walk) component and \(v_{it}\) is is a transitory MA(1) process. \(w_{it}\) and \(e_{it}\) are iid uncorrelated shocks.

The GMM approach to estimation involves calculating sample variances and autocovariances of \(y_{it}\) in levels and/or first differences and relating them to their population counterparts.

In levels, the population moments of interest are:
%
\begin{align}
	\Cov(y_{it}, y_{is}) &= \Cov(u_{it}, u_{is}) + \Cov(v_{it}, v_{is}) \\
	\Cov(u_{it}, u_{is}) &= \sigma_{u_{1}}^2 + (\min(t,s) - 1)\sigma_w^2 \\
	\Cov(v_{it}, v_{is}) &=\begin{cases}
		(1 + \theta^2)\sigma_w^2 & \text{if $t=s$} \\
		\theta\sigma_e^2 & \text{if $|t-s|=1$} \\
		0 & \text{otherwise}
	\end{cases}
\end{align}
%
where we assume we estimate the variance of the initial value of the permanent process as a separate parameter, \(\sigma_{u_{1}}^2\)

In first differences, the population moments of interest are:
%
\begin{align}
	\Cov(\Delta y_{it}, \Delta y_{is}) &=\begin{cases}
		\sigma_w^2 + 2(\theta^2 - \theta + 1)\sigma_e^2 & \text{if $t=s$} \\
		-(\theta - 1)^2\sigma_e^2 & \text{if $|t-s|=1$} \\
		-\theta\sigma_e^2 & \text{if $|t-s|=2$} \\ 	
		0 & \text{otherwise}
	\end{cases}
\end{align}
%
Notice that none of the moments in first differences depend upon the variance of the initial value of the permanent process, \(\sigma_{u_{1}}^2\), so we won't be able to estimate this.

\section{Estimation code}

\texttt{estimate\_income\_process\_Perm\_MA1.r} contains R code to simulate data and estimate the parameters of the wage process using GMM based on the moments above. \texttt{estimate\_income\_process\_Perm\_MA1.do} contains the corresponding Stata code. Each take a few minutes to run.

\end{document}